\chapter{Conclusion}
In this thesis the basic principles of electroweak baryogenesis and more specifically baryogenesis via leptogenesis were presented. It was shown that for matter antimatter asymmetry creating mechanisms in general the three conditions of baryon number violation, C and CP violation and a departure from the thermal equilibrium, the so called Sakharov conditions, have to be met. \newline \indent
Although the current Standard Model of particle physics could offer everything to meet this conditions in theory, recent experiments have shown that the mass of the Higgs boson is too high to induce an  electroweak phase transition of sufficiently strong first order or of first order at all. This phase transition of strong first order, however, is needed to ensure that the non-perturbative B violating processes happen outside of thermal equilibrium. \newline \indent
To circumvent this problem the Standard Model is expanded by heavy, right-handed partners of the nowadays observable light neutrinos. By assuming that neutrinos are Majorana particles, so that they are their own antiparticle, these heavy neutrinos could then decay CP violating into leptons and antileptons, respectively, violating lepton number in the process and creating an excess lepton number. For sufficiently high temperatures this lepton number can then be transformed into an excess baryon number by $B+L$ violating processes. This baryon number is then frozen out as soon as the temperature is low enough for the inverse decay to be strongly Boltzmann suppressed. This scenario is known as baryogenesis via leptogenesis. \newline \indent
In addition to this, the rate equations for leptogenesis and the coefficients appearing in the rate equations were obtained in the non-relativistic limit as well as in an expansion including terms of up to first order in relativistic corrections. These equations were then solved in order to compute the $B-L$ asymmetry in the so called strong washout regime, meaning $K>1$ and the non-relativistic approximation was compared with the relativistic corrections, clearly showing that in the strong washout regime the relative difference between these two is small and that the non-relativistic approximation is a valid one. Also the effect of using different statistics has been computed, showing a far greater influence than the relativistic corrections.  \newline \indent
Finally, it can be said that a first hint for verifying the model of leptogenesis would be by experimentally proving the Majorana character of neutrinos through processes like the neutrinoless double beta decay. Although no verified event of such a decay has been observed yet, there is still hope that neutrinos are Majorana particles simply because of the tremendous half-life of $10^{24}-10^{25}$ years\cite{Arnold:2016ezh}.

\chapter*{Acknowledgements}
I want to thank my supervisor Dr. Antonio Vairo for supporting and helping me throughout the whole time I worked on this thesis. \newline \indent
Also thanks to Dr. Mirco Wörmann from the Bielefeld University, whose input cleared up many problems regarding the implementation of the relativistic corrections to the numerical solutions of the rate equations.

