\section{Outline of baryogenesis}
Actually in 1967 the Sovjetian physicist Andrei Sakharov postulated the criteria, which have to be met in order for an excess of baryons over anti-baryons to be generated out a fully symmetrical universe.
\subsection{Sakharov Conditions}
As mentioned above there are three crucial properties of nature, the Sakharov conditions, wich are required to produce an net baryon number greater than zero. These three conditions are:
\begin{enumerate}
	\item B-violating process(es)
	\item C and CP violation
	\item Departure from or loss of thermal equilibrium
\end{enumerate}
For an general insight of these three conditions the first one will be skipped, since it is quite obvious, that in an totally symmetric universe there has to be at least one B-violating process in order to cause an inbalance in matter and antimatter. \newline
The general importance of the other two will be discussed in the following.
\subsubsection{C and CP-violation}
Charge conjugation (C), parity (P) and der combination (CP) are two or more specifically three basic symmetries of the universe. C symmetry states, that physical processes are the same, even after exchanging particles for their respective anti-particles, while P-symmetry guarantees invariance under the transformation $\vec{r}\rightarrow-\vec{r}$. CP symmetry then simply is an sequence of a C followed by a P transformation. \newline
To explain why C has to be violated for baryogenesis beeing possible, consider the B-violating reaction
\begin{equation*}
	X\rightarrow Y+B
\end{equation*}
with X and Y particles with B=0 and B representing the excess baryons. This reactions happens with a certain rate, which is, using C as a symmetry, just the same as the reation rate for the conjugate process.
\begin{equation}
	\Gamma(X\rightarrow Y+B)=\Gamma(\bar{X}\rightarrow \bar{Y}+\bar{B})
	\label{c-violation}
\end{equation}
Eq. \ref{c-violation} implies, that under C exactly the same amount of baryons and anti-baryons will be produced and therefore no excess baryons are left after the annihilaton. This means C must be violated. \newline
But additionally to this CP violation is essential for baryogenesis. To illustrate why, take a closer look at the also clearly B  violating X decay with its two channels:
\begin{align*}
	X\rightarrow q_Lq_L\\
	X\rightarrow q_Rq_R
\end{align*}
with q an arbritrary quark. The subscripts L and R denote the the left - or right-handedness chirality of the decay products. CP then effects each particle as follows
\begin{align*}
	X\underset{CP}{\longrightarrow}\bar{X}\\
	q_{L}\underset{CP}{\longrightarrow}\bar{q}_{R}\\
	q_{R}\underset{CP}{\longrightarrow}\bar{q}_{L}
\end{align*}
So CP doesn't just change matter for anti-matter, but the handedness of the particles as well. So if CP holds as a symmetry the consequences for the reaction rates are:
\begin{equation*}
	\Gamma(X\rightarrow q_Lq_L)=\Gamma(\bar{X}\rightarrow \bar{q}_R\bar{q}_R)\hspace{2cm}\Gamma(X\rightarrow q_Rq_R)=\Gamma(\bar{X}\rightarrow \bar{q}_L\bar{q}_L)
\end{equation*}
Adding these two results in 
\begin{equation}
\Gamma(X\rightarrow q_Lq_L)+\Gamma(X\rightarrow q_Rq_R)=\Gamma(\bar{X}\rightarrow \bar{q}_R\bar{q}_R)+\Gamma(\bar{X}\rightarrow \bar{q}_L\bar{q}_L)
\label{cp-violation}
\end{equation}
Eq. \ref{cp-violation} implies, that as long as there are as many particles X as anti-particles $\bar{\text{X}}$ in the initial state of the universe, which is just the starting point of the modell the Sakharov conditions try to describe, there can only be an asymmetry between left and right-handed particles be achieved, but that isn't a baryon asymmetry, which is clearly needed for baryogenesis. CP must be violated.\newline
So the bottom line here is, that the existence of B-violating processes is not sufficient for baryogenesis, but that there also has to be C and also CP violation, since without this kind of symmetry breaking any baryonic excess would be washed out by the corresponding C or CP conjugated process, as shown with the simple examples above. 
\subsubsection{Departure from thermal equilibrium}
The last condition to be met in order for baryogenesis to be achievable is that the the B, C and CP violating processes must occur outside the thermal equilibrium. To illustrate this we first consider the phase space distribution of a species X of quantum particles
\begin{equation}
	f(E_X)=\frac{1}{e^{\frac{E_X-\mu_X}{T}}\pm1}
	\label{distribution}
\end{equation}
The energy E$_X$ and the momentum $\vec{p}_X$ are related via the relativistic energy-momentum-relation $E^2=\vec{p}^2+m^2$. $\mu_X$ describes the chemical potential of the particle species X, which is an important quantity for describing thermal equilibrium states, as the chemical potentials of two species X and Y, which are in thermal equilibrium are related by $\mu_X=\mu_Y$ or for more species $\sum_i\mu_i=0$.\newline
Using eq. \ref{distribution} to compute the particle density of a certain particle species one gets 
\begin{equation*}
	n_X=g_X\int\frac{d^3p}{(2\pi)^3}\:f_X(E)
\end{equation*}
where g$_X$ denotes the number of inner degrees of freedom of X. \newline
Integrating this over phase space for non-relativistic particles, meaning T$\ll$m$_X$ yields
\begin{equation}
n_X=g_X\left(\frac{m_XT}{2\pi}\right)^\frac{3}{2}e^{-\frac{m_X-\mu_X}{T}}
\label{numerX}
\end{equation}
Analogously you get the number density for the corresponding anti-particle $\bar{\text{X}}$
\begin{equation}
	n_{\bar{X}}=g_{\bar{X}}\left(\frac{m_{\bar{X}}T}{2\pi}\right)^\frac{3}{2}e^{-\frac{m_{\bar{X}}-\mu_{\bar{X}}}{T}}
\label{numberantiX}
\end{equation}
Now suppose X and its anti-particle $\bar{\text{X}}$  with B$_X=-$B$_{\bar{X}}\neq0$ are in thermal equilibrium than the condition $\mu_X=\mu_{\bar{X}}$ holds. Comparing eq. \ref{numerX} and \ref{numberantiX} one sees, that the chemical %potential is the only property that could differ for particles and antiparticles. Now using the equilibrium condition for chemical one finally gets
\begin{equation}
	n_X=n_{\bar{X}}
	\label{thermalequ}
\end{equation}
Looking at eq \ref{thermalequ} it is quite obvious that even with B, C and CP violating any produced excess baryon number B will be washed out in equilibrium by other processes happening in equilibrium. \newline
This illustrates the final Sakharov Condition, that next to B, C and CP violation a departure from equilibrium is needed for a dynamic production of excess baryons.
\subsection{Baryogenesis in the Standard Modell}
\subsubsection{Electroweak baryogenesis}
\subsubsection{Failures of the SM}

%%% mode: latex
%%% TeX-master: "../Leptogenesis"
%%% End: