\section{Outline of baryogenesis}
Actually in 1967 the Sovjetian physicist Andrei Sakharov postulated the criteria, which have to be met in order for an excess of baryons over anti-baryons to be generated out a fully symmetrical universe.
\subsection{Sakharov Conditions}
As mentioned above there are three crucial properties of nature, the Sakharov conditions, wich are required to produce an net baryon number greater than zero. These three conditions are:
\begin{enumerate}
	\item B-violating process(es)
	\item C and CP violation
	\item Departure from or loss of thermal equilibrium
\end{enumerate}
For an general insight of these three conditions the first one will be skipped, since it is quite obvious, that in an totally symmetric universe there has to be at least one B-violating process in order to cause an inbalance in matter and antimatter. \newline
The general importance of the other two will be discussed in the following.
\subsubsection{C and CP-violation}
Charge conjugation (C), parity (P) and their combination (CP) are two or more specifically three basic symmetries of the universe. C symmetry states, that physical processes are the same, even after exchanging particles for their respective anti-particles, while P-symmetry guarantees invariance under the transformation $\vec{r}\rightarrow-\vec{r}$. CP symmetry then simply is an sequence of a C followed by a P transformation. \newline
To explain why C has to be violated for baryogenesis beeing possible, consider the B-violating reaction
\begin{equation*}
	X\rightarrow Y+B
\end{equation*}
with X and Y particles with B=0 and B representing the excess baryons. This reactions happens with a certain rate, which is, using C as a symmetry, just the same as the reation rate for the conjugate process.
\begin{equation}
	\Gamma(X\rightarrow Y+B)=\Gamma(\bar{X}\rightarrow \bar{Y}+\bar{B})
	\label{c-violation}
\end{equation}
Eq. \ref{c-violation} implies, that under C exactly the same amount of baryons and anti-baryons will be produced and therefore no excess baryons are left after the annihilaton. This means C must be violated. \newline
But additionally to this CP violation is essential for baryogenesis. To illustrate why, take a closer look at the also clearly B  violating X decay with its two channels:
\begin{align*}
	X\rightarrow q_Lq_L\\
	X\rightarrow q_Rq_R
\end{align*}
with q an arbritrary quark. The subscripts L and R denote the the left - or right-handedness chirality of the decay products. CP then effects each particle as follows
\begin{align*}
	X\underset{CP}{\longrightarrow}\bar{X}\\
	q_{L}\underset{CP}{\longrightarrow}\bar{q}_{R}\\
	q_{R}\underset{CP}{\longrightarrow}\bar{q}_{L}
\end{align*}
So CP doesn't just change matter for anti-matter, but the handedness of the particles as well. So if CP holds as a symmetry the consequences for the reaction rates are:
\begin{equation*}
	\Gamma(X\rightarrow q_Lq_L)=\Gamma(\bar{X}\rightarrow \bar{q}_R\bar{q}_R)\hspace{2cm}\Gamma(X\rightarrow q_Rq_R)=\Gamma(\bar{X}\rightarrow \bar{q}_L\bar{q}_L)
\end{equation*}
Adding these two results in 
\begin{equation}
\Gamma(X\rightarrow q_Lq_L)+\Gamma(X\rightarrow q_Rq_R)=\Gamma(\bar{X}\rightarrow \bar{q}_R\bar{q}_R)+\Gamma(\bar{X}\rightarrow \bar{q}_L\bar{q}_L)
\label{cp-violation}
\end{equation}
Eq. \ref{cp-violation} implies, that as long as there are as many particles X as anti-particles $\bar{\text{X}}$ in the initial state of the universe, which is just the starting point of the modell the Sakharov conditions try to describe, there can only be an asymmetry between left and right-handed particles be achieved, but that isn't a baryon asymmetry, which is clearly needed for baryogenesis. CP must be violated.\newline
So the bottom line here is, that the existence of B-violating processes is not sufficient for baryogenesis, but that there also has to be C and also CP violation, since without this kind of symmetry breaking any baryonic excess would be washed out by the corresponding C or CP conjugated process, as shown with the simple examples above. 
\subsubsection{Departure from thermal equilibrium}
The last condition to be met in order for baryogenesis to be achievable is that the the B, C and CP violating processes must occur outside the thermal equilibrium. To illustrate this we first consider the phase space distribution of a species X of quantum particles
\begin{equation}
	f(E_X)=\frac{1}{e^{\frac{E_X-\mu_X}{T}}\pm1}
	\label{distribution}
\end{equation}
The energy E$_X$ and the momentum $\vec{p}_X$ are related via the relativistic energy-momentum-relation $E^2=\vec{p}^2+m^2$. $\mu_X$ describes the chemical potential of the particle species X, which is an important quantity for describing thermal equilibrium states, as the chemical potentials of two species X and Y, which are in thermal equilibrium are related by $\mu_X=\mu_Y$ or for more species $\sum_i\mu_i=0$.\newline
Using eq. \ref{distribution} to compute the particle density of a certain particle species one gets 
\begin{equation*}
	n_X=g_X\int\frac{d^3p}{(2\pi)^3}\:f_X(E)
\end{equation*}
where g$_X$ denotes the number of inner degrees of freedom of X. \newline
Integrating this over phase space for non-relativistic particles, meaning T$\ll$m$_X$ yields
\begin{equation}
n_X=g_X\left(\frac{m_XT}{2\pi}\right)^\frac{3}{2}e^{-\frac{m_X-\mu_X}{T}}
\label{numerX}
\end{equation}
Analogously you get the number density for the corresponding anti-particle $\bar{\text{X}}$
\begin{equation}
	n_{\bar{X}}=g_{\bar{X}}\left(\frac{m_{\bar{X}}T}{2\pi}\right)^\frac{3}{2}e^{-\frac{m_{\bar{X}}-\mu_{\bar{X}}}{T}}
\label{numberantiX}
\end{equation}
Now suppose X and its anti-particle $\bar{\text{X}}$  with B$_X=-$B$_{\bar{X}}\neq0$ are in thermal equilibrium than the condition $\mu_X=\mu_{\bar{X}}$ holds. Comparing eq. \ref{numerX} and \ref{numberantiX} one sees, that the chemical %potential is the only property that could differ for particles and antiparticles. Now using the equilibrium condition for chemical one finally gets
\begin{equation}
	n_X=n_{\bar{X}}
	\label{thermalequ}
\end{equation}
Looking at eq \ref{thermalequ} it is quite obvious that even with B, C and CP violating any produced excess baryon number B will be washed out in equilibrium by other processes happening in equilibrium. \newline
This illustrates the final Sakharov Condition, that next to B, C and CP violation a departure from equilibrium is needed for a dynamic production of excess baryons. \newline
Interresting to note is, that there is quite an easy way of approximatelly determining if reactions take place in thermal equilibrium is by comparing the reaction rate with the expansion of universe, discribed by the Hubble constant H, which isn't actually a constant but changes with time. So if the relation 
\begin{equation}
	\Gamma\gtrsim H
	\label{rate_g_hubble}
\end{equation}
holds, the reactions take place fast enough for them to be in equilibrium. This can be made understandable, it is useful to look at this from the rest fram of the particles taking part in the reactions. Then the particles don't notice any expansion of the universe since they move and react to fast with each other, therefore the expansion doesn't really affect the equilibrium state. \newline
Otherwise if the reactions occur slower than the universe expands, so if 
\begin{equation}
	\Gamma<H
\end{equation}
is valid, than the expansions happens fast enough that particles get separated to far from each other, so they can't react anymore and the reactions fall out of equilibrium.
\subsection{Baryogenesis in the Standard Modell}
Although nowadays there are no records or experimental proofs of baryon number violating processes, that doesn't mean there is a need for physics outside the Standard Modell (SM) of particle physics, at least on a qualitative level.
\subsubsection{Electroweak baryogenesis}
As it turns out the electroweak part of the SM with its SU(2)$_L\times$U(1)$_Y$ symmetry groups suits best for describing baryogenesis. \newline
\paragraph{C and CP violation}
It is already proven theoretically und experimentally by numerous well-known experiments, like for example the Wu experiment in 1956, that C symmetry is maximally violated by the weak interaction in the leptonic as well as in the hadronic sector. As shown by Kobayashi and Maskawa through expanding the Cabibbo hypothesis and experimentally confirmed, weak interactions in the hadronic sector also violate CP invariance, which manifests as an complex phase in the CKM quark mixing matrix. In the leptonic sector however the CP violation through a complex phase only got postulated in the PMNS neutrino mixing matrix to try to descripe neutrino oscillations, but this phase still needs to be measured.\newline
Nevertheless the elektroweak part of the SM, more precise the weak interactions, since electromagnetism doesn't violate C or even P, satisfies at least one of the three Sakharov conditions. The question now is, how modern particle physics fulfills the other conditions as well. \newline
\paragraph{B violation}
Although the first Sakharov condition, the necessity of baryon number violating processes, seems to be the most obvious, the way these are realised in the SM is a bit more difficult than it seems. \newline
Since at the first look the baryonic and, since it is going to play an important role during the following discussion, the leptonic current are  conserved
\begin{align}
	\partial^\mu J_\mu^B=0
	\label{Bcurrent}
	\\
	\partial^\mu J_\mu^L=0
	\label{Lcurrent}
\end{align}
one would assume there is no way the SM could produce an baryon asymmetry. However, by considering quantum fluctuation meaning orders higher than just tree level one finds, that the currents for the left- and right-handed parts f$_L$ and f$_R$ respectively, where stands for quarks and leptons equally, aren't conserved and not the same ["'CP violation and baryogenesis`']
\begin{align}
	\partial^\mu\bar{f}_L\gamma_\mu f_L&=-c_L\frac{g^2}{32\pi^2}F^a_{\mu\nu}\tilde{F}^{a\mu\nu}
	\label{l_chiralty_current}
	\\
	\partial^\mu\bar{f}_R\gamma_\mu f_R&=+c_R\frac{g^2}{32\pi^2}F^a_{\mu\nu}\tilde{F}^{a\mu\nu}
	\label{r_chiralty_current}
\end{align}
where g denotes the gauge coupling, F$^{a\mu\nu}$ the field tensor, $\tilde{F}^{a\mu\nu}$ the dual field tensor and c$_L$ and c$_R$ depend on the representation of f$_L$ and f$_R$.This behaviour of the currents at quantum levels is known as Adler-Bell-Jackiw or chiralty anomaly. 
Since SU(2)$_L$ gauge boson only couples with left-handed particles c$_R^W$=0, while the U(1)$_Y$ gauge boson couples to both hadednesses, but with different strength, therefore c$_R^Y\neq$c$_L^Y$. Although this section only focuses on electroweak baryogenesis, it is mentionable that with the SU(3)$_c$ gauge bosons of the strong interactions don't produce any chiralty anomaly because they couple with left as well as right-handed particles with the same strenght, so c$_R^c=$c$_L^c$ and both currents in \eqref{l_chiralty_current} and \eqref{r_chiralty_current} cancel each other out in the case of strong interactions. \newline
Putting this and eqautions \ref{Bcurrent} - \ref{r_chiralty_current} together, gives a pretty interesting result
\begin{equation}
\partial^\mu J_\mu^B=\partial^\mu J_\mu^L=\frac{n_F}{32\pi^2}\left(-g_w^2W^a_{\mu\nu}\tilde{W}^{a\mu\nu}g'^2G^a_{\mu\nu}\tilde{G}^{a\mu\nu}\right)
\label{B-L}
\end{equation}
with W$^{a\mu\nu}$ and B$^{a\mu\nu}$ the field strenght tensors of the SU(2)$_L$ and U(1)$_Y$ gauge groups and n$_F$=3 the number of particle families. \newline
Analyzing eq. \ref{B-L} one easily figures out, that although baryon and lepton number are not conserved separatedly the difference B-L of these numbers is very well conserved.
\subsubsection{Failures of the SM}

%%% mode: latex
%%% TeX-master: "../Leptogenesis"
%%% End: