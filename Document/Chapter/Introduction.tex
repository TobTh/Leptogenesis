\section{Introduction}
[...]
One way to describe the observed baryonic asymmetry is by a postulating, that the universe has been in an asymmetric state just from the beginning and that the matter and antimatter is concentrated in big domains throughout the universe, which come into contact just at their outer borders. Techincally there is no reason for the universe not to have started in an asymmetric state, in that case one would measure high gamma rates due to the matter-antimatter-annihilation right between these distinct regions. \newline
Since there is no kind of such radiation seen, patches of diffrent kinds of matter have to be as big as the presently observable universe. Because this doesn't seems very plausible, so the baryonic asymmetry had to arise dynamically from an universe where matter and antimatter existed in the same amounts. \newline
[...]\newpage
%%% mode: latex
%%% TeX-master: "../Laser"
%%% End: