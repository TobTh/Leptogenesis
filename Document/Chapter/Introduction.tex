\chapter{Introduction}
One of todays greatest unsolved mystery is the observed asymetry between baryons and antibaryons in the known universe. A useful way to quantify this asymmetry is by using
\begin{equation}
	\eta=\frac{n_b-n_{\bar{b}}}{n_\gamma}=\frac{n_B}{n_\gamma},
	\label{eq:asymmetry}
\end{equation}
where $n_b$ and $n_{\bar{b}}$ are the number densities of baryons and antibaryons respectively, $n_B=n_b-n_{\bar{b}}$ and $n_\gamma$ the photon density that can be calculated using thermodynamics for massless bosons. Using the fact that in the visible universe nearly no antimatter is observe leads to $n_b-n_{\bar{b}}\simeq n_b$ and therefore
\begin{equation}
\eta\simeq\frac{n_b}{n_\gamma}
\end{equation}\newline \indent
Although measuring the exact value of $\eta$ is impossible because this would include counting all baryons and antibaryons in the whole universe there are two ways obtaining a range of values fot $\eta$, namely using the model of Big Bang Nucleosynthesis (BBN) and the Cosmic Microwave Background (CMB)\cite{Sarkar:2002er}. In the first case the primordial abundances of the four stable isotopes formed during BBN, namely D, $^3$He, $^4$He and $^7$Li, that are observed, are compared to the $\eta$-dependent values of these abundances predicted by the theoretical BBN model, resulting in 
\cite[Eq. (1.25)]{Biondini:2016hhn}
\begin{equation}
	4.7\times10^{-10}\leq\eta\leq6.5\times10^{-10}.
\end{equation}
On the other hand a more precise way of probing the baryon asymmetry is by using the CMB or more precisely any anisotropies showing in the otherwise isotropic spectrum. Analyzing these anisotropies gives a pretty precise result of \cite[Eq. (1.26)]{Biondini:2016hhn}
\begin{equation}
\eta=(6.1\pm0.16)\times10^{-10},
\label{eq:eta_value}
\end{equation}
which coincides well with the range for $\eta$ given above.\newline \indent
Now the most intuitive way to introduce this asymmetry to the universe is to set it as a initial condition, meaning by assuming the universe already started in an asymmetric state. This initial abundance, however, would be diluted so much by the inflation of the universe described by the in many other ways successful big bang theory, leaving behind no significant baryon asymmetry. One could also argument, that the universe is strictly symmetric considering baryon number, but that matter and antimatter are concentrated in big domains throughout the universe, which come into contact just at their outer borders. This would lead to big gamma bursts at their borders due to annihilation of matter and antimatter that would greatly disturb the isotropic nature of the CMB. Since no kind of such distortions are observed it is safe to say that such big clusters of antimatter do not exist or they have to be at least as big as the presently observable universe\cite{Cline:2006ts}. Interesting to note is that even with a completly symmetric universe there would still be some baryons and antibaryons present, their number density, however, would be extremely Boltzmann suppressed resulting in
\begin{equation}
	\frac{n_b}{n_\gamma}\approx	\frac{n_{\bar{b}}}{n_\gamma}\approx10^{-18},
\end{equation}
which is eight orders of magnitude below the observed value given in \eqref{eq:eta_value}.\newline \indent
Putting all this together leads to the necessity of a theory in which the universe is initially matter-antimatter symmetric and the asymmetry is generated dynamically over time. This can happen via baryon number violating processes that produce the baryon asymmetry and therefore induce the so called baryogenesis. On the other hand a lepton number asymmetry can be produced by the CP-violating decay of heavy, sterile neutrionos that will than be transformed into a baryon number asymmetry by baryon plus lepton number violating processes. This whole process is called baryogenesis through leptogenesis. \newline \indent
The outline of this thesis then is the following: Chapter 2 will introduce the basic principles of baryogenesis and how these principles can and cannot be realised in the current Standard Model (SM) of particle phyics. Chapter 3 will then focus on how to extend the SM in order to sucessfully adopt a scenario in which the baryon asymmetry is produced by baryogenesis through leptogenesis. Chapter 4 then quantitatively describes leptogenesis using Boltzmann equations in a non-relativistic regime and takes lowest order relativistic and radiative corrections into account. Chapter 5 finally shows the numerical results of the rate equations introduced in chapter 4 as well as the effect of the relativistic% and radiative
corrections as well as the importance of the usage of the correct statistics.