\chapter{Outline of leptogenesis}
As stated in the section before the SM alone isn't quite enough to describe the observed baryon abundance, so the SM has to be expanded to that effect that it can describe such phenomena. \newline
Although are efforts made to explain direct baryogenesis using GUT theories, there is an much more favored alternative, namely the so called barygenesis via leptogenesis, what this and the following section will be about. 
\section{Expandig the SM}
There are experimental reasons why the SM doesn't tell the whole story about our universe, namely the results of neutrino oscillation experiments. Before the discovery of these neutrino oscillations it was accepted that neutrinos are massless and therefore their left-handedness is well defined. But being able to oscillate between different flavours implies that neutrinos aren't massless and therefore are not purely left-handed and even more so that right-handed neutrinos exist. The easisest way to implement right-handed neutrinos into the SM would be to show, that neutrinos are so-called Majorana particles, which, in contrast to Dirac particles, are their own anti-particles. This would mean that the right-handed neutrinos are right-handed antineutrinos at the same time, but it was already shown that latter exist. Theoretically the way to describe Majorana masses would be to exchange the usual mass term including the Higgs field for a Majorana mass terms, that can be written in the following way \cite[p. 18]{Taanila:2008}, to the SM Lagrangian.
\begin{equation*}
	\mathcal{L}_M=-\frac{1}{2}\overline{\Psi^C}M^M\Psi
	\label{eq:majorana}
\end{equation*}
where the superscript C stands for the charge conjugated neutrino field defined by
\begin{equation*}
	\Psi^C\equiv C\gamma_0\Psi^*
\end{equation*}
with the marix C, which is dependend on the representation of the gamma matrices. The Majorana mass M$^M$ is an n$_F\times$n$_F$ matrix with n$_F$ again the number of particle families. \newline
Using the representation of the U(1)$_Y$ symmetry group given in the previous section, it can easily be seen that in general by using a mass term like in \ref{eq:majorana} to the SM Lagrangian this symmetry is no longer viable.
\begin{equation*}
	\overline{\Psi^C}\Psi \overset{U(1)Y}{\longrightarrow}\overline{(e^{i\frac{Y}{2}\alpha}\Psi)^C}e^{i\frac{Y}{2}\alpha}\Psi=\overline{e^{-i\frac{Y}{2}\alpha}\Psi^C}e^{i\frac{Y}{2}\alpha}\Psi=e^{i\frac{Y}{2}\alpha}\overline{\Psi^C}e^{i\frac{Y}{2}\alpha}\Psi\neq	\overline{\Psi^C}\Psi
\end{equation*}
The relation above shows, that by using Majorana masses of particles with hypercharge Y$\neq$0, like the left-handed neutrinos, one cannot preserve the SM Lagrangians SU(2)$_L\times$U(1)$_Y$ symmetry. 
\newline
That being said right-handed Dirac mass terms for neutrinos have to be added to the Lagrangian in order for it to still preserve its SU(2)$_L\times$U(1)$_Y$ symmetry. Since, like all other right-handed particles in the SM, the right-handed neutrinos form a SU(2)$_Y$ singlet, which we will call N. Also using the isospin conjugate of the Higgs doublet 
\begin{equation*}
	\tilde{\phi}\equiv i\sigma^2\phi^*
\end{equation*}
the Yukawa term can be written as
\begin{equation}
	\mathcal{L}_{\text{N,Yuk}}=h_{ij}\overline{N_i}\tilde{\phi}^\dagger l_j +h_{ij}^* \overline{l_i}\tilde{\phi} N_j
	\label{eq:Yukterm}
\end{equation}
This interaction term will be of analyzed in more detail in Appendix A.1. \newline
The h$_{ij}$ describe the Yukawa couplings and the l$_i$ the left-handed lepton SU(2) doublets of the Standard modell. It can be shown, that this additional term doesn't violate the symmetries of the SM Lagrangian. \newline
However, as explained above adding left-handed Majorana neutrino mass terms to the SM Lagrangian breaks its symmetry, but since right-handed neutrinos have to be added anyways one can also try to add a right-handed Majorana mass term, too.
\begin{equation}
\mathcal{L}_{N,M}=-\frac{1}{2}\overline{N^C}M^MN
\label{eq:neutrino_majorana}
\end{equation}
The mass term in equation \ref{eq:neutrino_majorana} however doesn't violate the Lagrangian's SU(2)$_L\times$U(1)$_Y$ symmetry, because for right-handed neutrinos T=T$^3$=Y=0 and therefore the transformations given in the previous section become the trivial identity transfomation. Anyways, the Dirac Lagrangian, so the SM Langrangian without any Majorana mass terms, is obviously invariant under any U(1) transformation, not only under U(1)$_Y$ transformations. The Majorana mass terms on the other hand are only invariant under the exactly this U(1)$_Y$, especially only for particles with T=T$^3$=Y=0 like the right-handed neutrinos while violating other U(1) symmetries. And according to the Noether theorem, every symmetry of a theory results in a conserved current or quantum number, so breaking the U(1) symmetry not assigned to the hypercharge by using Majorana mass terms one certain quantum number, in this case breaking the following U(1)$_l$ symmetry results in a non-conservation of the lepton number.
\begin{equation}
	\Psi\overset{U(1)_L}{\longrightarrow}e^{i\ell\alpha(x)}\Psi
\end{equation}
with the lepton number $\ell$ of the field $\Psi$.
This seems rather obvious because if Majorana particles are particles and anti-particles at the same time one cannot assing them a distinct lepton number and therefore it is not conserved. \newline
Finally, after putting the Dirac and Majorana mass terms together one ends up with \cite[p. 21]{Taanila:2008}
\begin{equation}
	\mathcal{L}_{M+D}=\left(\overline{\nu^C_L},\overline{N}\right)	\left(\begin{array}{cc}M^M_L&m^D\\m^D&M^M_R\end{array}\right)	\left(\begin{array}{c}\nu_L\\N^C\end{array}\right)+h.c.
	\label{eq_majorana_dirac}
\end{equation}
The matrix m$^D$ contains the masses of the Dirac neutrinos, so the neutrinos found in nature up until now, while M$^M_L$ and M$^M_R$ describe the masses of the left as well as the right-handed Majorana neutrinos. All of these matrices are of the dimension n$_F\times$n$_F$ with n$_F$ the number of neutrino flavours. Also, as explained above, since it isn't possible to introduce left-handed Majorana neutrinos to the SM without violating its fundemental symmetry M$^M_L$ has to be equal to zero.
\section{The seesaw Mechanism}
Although the addition of neutrino masses can be described using the mass term \ref{eq_majorana_dirac} or rather
\begin{equation}
\mathcal{L}_{M+D}=\left(\overline{\nu^C_L},\overline{N}\right)	\left(\begin{array}{cc}0&m^D\\m^D&M^M_R\end{array}\right)	\left(\begin{array}{c}\nu_L\\N^C\end{array}\right)+h.c.
\label{eq_majorana_dirac_zero}
\end{equation}
there is still a problem, namely why the neutrino masses are many orders of magnitude smaller than those of the other SM particles. This however can be described using the so called seesaw mechanism. In this discussion only the so called type I seesaw mechanism will be presented. By doing so the following two assumtions have to be made:
\begin{enumerate}
	\item The Dirac masses arise directly from the Higgs mechanism that gives mass to all SM particles, introducing the electroweak mass scale of order $\sim10^2-10^3$ GeV.
	\item The Majorana masses are much bigger than the Dirac masses, m$^D\ll$ M$^M$. This mass scale arises from GUT's and is of order $\sim10^{10}-10^{16}$ GeV.
\end{enumerate}
The subscript R will be dropped from now on since there is no non-zero left-handed Majorana mass and therefore no further distinction is needed.\newline
Now, after diagonalising the mass matrix in \ref{eq_majorana_dirac_zero} \cite[pp. 2-3]{Lindner:2001hr}, one gets two mass eigenvalues, in particular
\begin{align*}
	M_1&\backsimeq-m^D{\left(M^M\right)}^{-1}{\left(m^D\right)}^{T}\\
	M_2&\backsimeq M^M
\end{align*}
or for just one neutrino family
\begin{align}
	M_1&\backsimeq -\frac{m_D^2}{M^M}
	\label{eq:light_neutrino}
	\\
	M_2&\backsimeq M^M
	\label{eq:heavy_neutrino}
\end{align}
The negative sign for M$_1$ comes from the fact that these are just the eigenvalues of the mass matrix given in \ref{eq_majorana_dirac_zero}, the physical masses are the absolute values of these eigenvalues. \newline
Finding the corresponding eigenstates for each eigenvalue one gets that the eigenstate associated with M$_1$ is $\nu$, the observable, left-handed light neutrino. One can now easily see that the smallness of the neutrino masses compared to those of all the other SM particles comes from the assumption m$^D\ll$ M$^M$. On the other hand however the eigenstate appendant to $M_2$ is N, the newly added, right-handed heavy neutrino, that will play a crucial role in leptogenesis. The heavy neutrino mass being mostly governed by the Majorana mass implies that the right-handed neutrinos are Majorana particles, which means they are their own anti particles.\newline
Interesting to note as well is how these two masses behave under finetuning. It is quite obvious from equation \ref{eq:light_neutrino} that by raising the large mass scale and as a consequence thereof raising the mass of the heavy neutrino the mass of the light neutrinos gets even lower and vice versa, hence the name seesaw mechanism. 
\section{Leptogenesis and the Sakharov conditions}
After the necessary expansion of the SM was performed in the previous section, this section will focus on how the right-handed, heavy neutrinos are able to produce a net, non-zero baryon number, that is how the Sakharov conditions can be fullfilled using this exapanded SM. \newline
The key ingredient for baryogenesis via leptogenesis is the decay of the heavy, right-handed neutrinos introduced above, that is described by the Yukawa interaction in \ref{eq:Yukterm}.The Feynman diagrams for both decay channels are depicted in figure \ref{fig:N-decay}. The right-handed neutrinos being Majorana particles as a result of the seesaw mechanism they don't preserve lepton number and because of this they can decay into leptons as well as anti leptons, as it can be seen in figure \ref{fig:N-decay}.
\begin{figure}[H]
	\begin{equation*}
	\feynmandiagram [baseline=(d.base), horizontal=d to b] {
		b -- [scalar, edge label=\(\overline{\phi}\)] a,
		b-- [fermion, edge label =\(\ell\)] c,
		d   -- [edge label=N] b}; 
	\qquad \text{or} \qquad
	\feynmandiagram [baseline=(d.base), horizontal=d to b] {
		b -- [scalar, edge label = \(\phi\)] a,
		b-- [anti fermion, edge label=\(\overline{\ell}\)] c,
		d  --[edge label=N] b  }; 
	\end{equation*}
	\caption{Feynman diagrams for the N-decay}
	\label{fig:N-decay}
\end{figure}
\subsubsection{B violation}
The B violation in the frame of leptogenesis is achieved in the same way as in direct electorweak baryogenesis via the B+L violating sphaleron processes. Because of the eventually by N decays produced net lepton number this means that the lepton abundance is converted into a baryon abundance by these processes.
\subsubsection{C and CP violation}
As in direct baryogenesis C violation is already maximally violated SM electroweak interaction, so this is also the case in this slightly extended model.
The CP violation for the N$_1$ decay however is a bit more complicated, since for it to be calculated one has to take the one-loop Feynman diagrams additional to the tree level decay into account. The interference between these diagramms gives than rise to the CP violation needed for successfull baryogenesis. These one-loop diagrams are depicted in figure \ref{fig:N_loop}.
\begin{figure}[H]
	\begin{equation*}
	\feynmandiagram [baseline=(d.base), horizontal=a to t1] 
	{
		
		a  -- [edge label=\(N_i\)] t1 --[anti fermion,edge label=\(\ell\)] t2 --[edge label=\(N_j\)] t3 -- [scalar,edge label=\(\phi\)]t1, t2 -- [scalar, edge label=\(\phi\)] p2,
		t3 -- [fermion, edge label=\(\ell\)] p1 ,
		
	};
	\qquad +\qquad
	\feynmandiagram [baseline=(d.base), horizontal=b to c] 
	{ 
		a --[edge label =\(N_i\)]b
			-- [half left, looseness=1.5, edge label = \(\ell\text{,}\overline{\ell}\)] c
			-- [scalar, half left, looseness=1.5,edge label = \(\phi\)] b, 
		c -- [edge label = \(N_j\)]d --[scalar, edge label =\(\phi\)] e,
		d--[fermion, edge label =\(\ell\)]f
	
	};
	\end{equation*}
	\caption{One-loop diagrams for the N-decay}
	\label{fig:N_loop}
\end{figure}
\noindent
By comparing this figure to \cite[Fig. 5.1]{Davidson:2008bu} one sees that in figure \ref{fig:N_loop} one diagramm is missing. This missing diagram hoever doesn't contribute to the CP violation and therefore it was neglected. The distinction between N\textsubscript{i} and N\textsubscript{j} has to be made since the virtual neutrinos trasmissioned during these processes aren't of the same flavour as the neutrino in the initial state. This means that in order for CP to be violated there has to exist at least an extra neutrino next to the lightest one. \newline
This being said one can calculate the CP asymmetry\cite[pp. 24ff.]{Davidson:2008bu}, which is in general defined as
\begin{equation}
	\epsilon=\frac{\Gamma(N_1\rightarrow\overline{\phi}\ell)-\Gamma(N_1\rightarrow\phi\overline{\ell})}{\Gamma(N_1\rightarrow\overline{\phi}\ell)+\Gamma(N_1\rightarrow\phi\overline{\ell})}
	\label{eq:CP_violation}
\end{equation}
by interfering the tree level decay with the one-loop decays. This results in \cite[p. 26]{Davidson:2008bu}
\begin{equation}
	\epsilon\gtrsim10^{-6}
	\label{eq:CP_value}
\end{equation}
Although this value is rather small it isn't zero and therefore the CP symmetry is violated by the decay of heavy right-handed Majorana neutrinos. 
\subsubsection{Deviation from the equilibrium}
The last condition that has to be met for successfull baryogenesis is that the decay of the heavy neutrinos must occur outside of equilibrium. For high enough temperatures, namely for T $\gtrsim$ M\textsubscript{1}, the decay of the neutrino is in equilibrium with the inverse decay $\phi\ell\rightarrow N$ and no net lepton number can be produced even if the other two conditions hold as shown in sec. 2.1.2. Even if during inflation an abundance of N was produced the lepton asymmetry would washed out by equilibrium processes as soon as the temperature rises up to at least the mass of the heavy neutrino during the reheating phase of the ealrly universe. \newline
However if the temperature drops below M$_1$ the heavy neutrino inverse decay is exponentially surpressed by a Boltzmann factor because with falling temperature it becomes exponentially more unprobable for a lepton and Higgs to have enough energy to form a heavy neutrino, while the neutrinos themselves can still decay into lepton and Higgs. Because of this the equilibrium density of the neutrinos is also heavily Boltzmann surpressed and if the decay rate is small enough the actual neutrino density can be greater than the equilibrium density and stay greater for an considerable amount of time. This means that decays happening before the actual neutrino density converges to the equilibrium density actually happen outside of equilibrium and a net lepton number can be produced that, because of surpression of inverse decays, won't be washed out and as a consequence thereof will be transformed into a baryon abundance through the B+L violating electroweak sphaleron processes, that will still be active as the heavy neutrino mass at least of order of the upper limit of the temperature range in which sphaleron processes act. \newline
The requirement for the neutrino decay to be slow enough to sustain its out-of-equilibrium state is the following \cite[p. 30]{Taanila:2008}.
\begin{equation}
	\Gamma_D<H
	\label{eq:out_of_eq}
\end{equation}
$\Gamma_D$ denotes the total decay rate of the neutrinos while H is the expansion rate of the universe. This relation is equivalent to the one given in \ref{eq:rate_s_hubble} as a requirement for processes to be out of equilibrium. 