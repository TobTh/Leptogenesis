\section{Outline of leptogenesis}
As stated in the section before the SM alone isn't quite enough to describe the observed baryon abundance, so the SM has to be expanded to that effect that it can describe such phenomena. \newline
Although are efforts made to explain direct baryogenesis using GUT theories, there is an much more favored alternative, namely the so called barygenesis via leptogenesis, what this and the following section will be about. 
\subsection{Expandig the SM}
There are experimental reasons why the SM doesn't tell the whole story about our universe, namely the results of neutrino oscillation experiments. Before the discovery of these neutrino oscillations it was accepted that neutrinos are massless and therefore their left-handedness is well defined. But being able to oscillate between different flavours implies that neutrinos aren't massless and therefore are not purely left-handed and even more so that right-handed neutrinos exist. The easisest way to implement right-handed neutrinos into the SM would be to show, that neutrinos are so-called Majorana particles, which, in contrast to Dirac particles, are their own anti-particles. This would mean that the right-handed neutrinos are right-handed antineutrinos at the same time, but it was already shown that latter exist. Theoretically the way to describe Majorana masses would be to exchange the usual mass term including the Higgs field for a Majorana mass terms, that can be written in the following way \cite[pp. 18]{Taanila:2008}, to the SM Lagrangian.
\begin{equation*}
	\mathcal{L}_M=-\frac{1}{2}\overline{\Psi^C}M^M\Psi
	\label{eq:majorana}
\end{equation*}
where the superscript C stands for the charge conjugated neutrino field defined by
\begin{equation*}
	\Psi^C\equiv C\gamma_0\Psi^*
\end{equation*}
with the marix C, which is dependend on the representation of the gamma matrices. The Majorana mass M$^M$ is an n$_F\times$n$_F$ matrix with n$_F$ again the number of particle families. \newline
Using the representation of the U(1)$_Y$ symmetry group given in the previous section, it can easily be seen that in general by using a mass term like in \ref{eq:majorana} to the SM Lagrangian this symmetry is no longer viable.
\begin{equation*}
	\overline{\Psi^C}\Psi \overset{U(1)Y}{\longrightarrow}\overline{(e^{i\frac{Y}{2}\alpha}\Psi)^C}e^{i\frac{Y}{2}\alpha}\Psi=\overline{e^{-i\frac{Y}{2}\alpha}\Psi^C}e^{i\frac{Y}{2}\alpha}\Psi=e^{i\frac{Y}{2}\alpha}\overline{\Psi^C}e^{i\frac{Y}{2}\alpha}\Psi\neq	\overline{\Psi^C}\Psi
\end{equation*}
The relation above shows, that by using Majorana masses of particles with hypercharge Y$\neq$0, like the left-handed neutrinos, one cannot preserve the SM Lagrangians SU(2)$_L\times$U(1)$_Y$ symmetry. 
\newline
That being said right-handed Dirac mass terms for neutrinos have to be added to the Lagrangian in order for it to still preserve its SU(2)$_L\times$U(1)$_Y$ symmetry. Since, like all other right-handed particles in the SM, the right-handed neutrinos form a SU(2)$_Y$ singlet, which we will cal N. Also using the isospin conjugate of the Higgs doublet 
\begin{equation*}
	\tilde{\phi}\equiv i\sigma^2\phi^*
\end{equation*}
the Yukawa term can be written as
\begin{equation}
	\mathcal{L}_{\text{N,Yuk}}=h_{ij}\overline{N_i}\tilde{\phi}^\dagger l_j +h_{ij}^* \overline{l_i}\tilde{\phi} N_j
	\label{eq:Yukterm}
\end{equation}
This interaction term will be of interest in the following section. \newline
The h$_{ij}$ describe the Yukawa couplings and the l$_i$ the left-handed lepton SU(2) doublets of the Standard modell. It can be shown, that this additional term doesn't violate the symmetries of the SM Lagrangian. \newline
However, as explained above adding left-handed Majorana neutrino mass terms to the SM Lagrangian breaks its symmetry, but since right-handed neutrinos have to be added anyways one can also try to add a right-handed Majorana mass term, too.
\begin{equation}
\mathcal{L}_{N,M}=-\frac{1}{2}\overline{N^C}M^MN
\label{eq:neutrino_majorana}
\end{equation}
The mass term in equation \ref{eq:neutrino_majorana} however doesn't violate the Lagrangian's SU(2)$_L\times$U(1)$_Y$ symmetry, because for right-handed neutrinos T=T$^3$=Y=0 and therefore the transformations given in the previous section become the trivial identity transfomation. Anyways, the Dirac Lagrangian, so the SM Langrangian without any Majorana mass terms, is obviously invariant under any U(1) transformation, not only under U(1)$_Y$ transformations. The Majorana mass terms on the other hand are only invariant under the exactly this U(1)$_Y$, especially only for particles with T=T$^3$=Y=0 like the right-handed neutrinos while violating other U(1) symmetries. And according to the Noether theorem, every symmetry of a theory results in a conserved current or quantum number, so breaking the U(1) symmetry not assigned to the hypercharge by using Majorana mass terms one certain quantum number, in this case the lepton number is not preserved. This seems rather obvious because if Majorana particles are particles and anti-particles at the same time one cannot assing them a distinct lepton number and therefore it is not conserved. \newline
Finally, after putting the Dirac and Majorana mass terms together one ends up with \cite[pp. 21]{Taanila:2008}
\begin{equation}
	\mathcal{L}_{M+D}=\left(\overline{\nu^C_L},\overline{N}\right)	\left(\begin{array}{cc}M^M_L&m^D\\m^D&M^M_R\end{array}\right)	\left(\begin{array}{c}\nu_L\\N^C\end{array}\right)+h.c.
\end{equation}
\subsection{The See-Saw Mechanism}
\subsection{Leptogenesis and the Shakarov Conditions}